\documentclass[12pt]{article}

\usepackage[utf8]{inputenc}
\usepackage[russian]{babel}
\usepackage[margin=2cm, paper=a4paper]{geometry}
\usepackage{amsthm,amsmath,amsfonts}
\usepackage{mathtools}
\usepackage{framed}
\usepackage{gensymb}
\usepackage{hyperref}
\usepackage{marvosym} % For Neptune.
\usepackage{wrapfig}
\usepackage{tabularx}
\usepackage{subcaption}
\usepackage[margin=10pt,font=small,labelfont=bf,labelsep=endash]{caption}
\usepackage{enumitem}
\usepackage{graphicx}

\author{Мумладзе Максимелиан}
\title{Домашнее задание №3}
\date{\today}

\numberwithin{equation}{section} % just for today

\DeclareMathOperator{\mat}{Mat}


\newtheorem{thm}{Теорема}[section]
\theoremstyle{definition}\newtheorem{defi}{Определение}


\newcounter{taskn}[section]
\newenvironment{task}[1]{%
\begin{framed}
  \noindent
  \underline{\bfseries Задача \stepcounter{taskn}\thetaskn. <<#1>>}
  \setlength{\parindent}{0cm}
}{%
\end{framed}
}

\newenvironment{note_enum}{%
\begin{framed}
  
  \begin{small}
  \begin{enumerate}[wide, labelwidth=!, labelindent=0pt]
    \vspace{-1pc}
}{%
  \end{enumerate}
  \end{small}
  \vspace{-1pc}
\end{framed}
}


\newcolumntype{L}[1]{>{\hsize=#1\textwidth%
\raggedright\arraybackslash}X}%
\newcolumntype{R}[1]{>{\hsize=#1\textwidth%
\raggedleft\arraybackslash}X}%
\newcolumntype{C}[1]{>{\hsize=#1\textwidth%
\centering\arraybackslash}X}

%- - - - - - - - - - - - - - - - - - - - - - - - - - - - - - - - - -  - - - - -
\begin{document}
\maketitle

\section{Краткий обзор}

\begin{wrapfigure}{R}{0.4\textwidth}
  \vspace{-1pc}
  \center \includegraphics[width = 0.35\textwidth]{img/Neptune}
  \caption{Нептун с <<Вояджера-2>>.}
\end{wrapfigure}

\textbf{Нептун}~--- восьмая и самая дальняя от Земли планета Солнечной системы. Нептун также является четвёртой по диаметру и третьей по массе планетой. Масса Нептуна в $17.2$ раза, а диаметр экватора в $3.9$ раза больше земных. Наглядное сравнение можно увидеть на рис.\,\ref{pic:earth_and_neptune}. Название планеты было предложено русско--немецким астрономом Василием Струве\footnote{\href{http://articles.adsabs.harvard.edu/full/seri/AN.../0025/0000164.000.html}{<<Second report of proceedings in the Cambridge Observatory relating to the new Planet (Neptune)>>}}. Её астрономический символ \Neptune ~--- стилизованная версия трезубца Нептуна. С момента открытия и до 1930 года Нептун оставался самой далёкой от Солнца известной планетой. После открытия Плутона Нептун стал предпоследней планетой, за исключением 1979--1999 годов, когда Плутон находился внутри орбиты Нептуна. Однако исследование пояса Койпера в 1992 году привело к обсуждению вопроса о том, считать ли Плутон планетой или частью пояса Койпера. В 2006 году Международный астрономический союз принял новое определение термина <<планета>> и классифицировал Плутон как карликовую планету, и, таким образом, вновь сделал Нептун последней планетой Солнечной системы.


\begin{wrapfigure}{L}{0.4\textwidth}
  \vspace{-1pc}
  \center \includegraphics[width = 0.4\textwidth]{img/Neptune_Earth_size_comparison}
  \caption{Нептун и Земля.}
  \label{pic:earth_and_neptune}
\end{wrapfigure}

Обнаруженный 23 сентября 1846 года, Нептун стал первой планетой, открытой благодаря математическим расчётам, а не путём регулярных наблюдений. Обнаружение непредвиденных изменений в орбите Урана породило гипотезу о неизвестной планете, гравитационным возмущающим влиянием которой они и обусловлены. Нептун был найден в пределах предсказанного положения. Вскоре был открыт и его спутник Тритон, однако остальные $13$ спутников, известные ныне, были неизвестны до XX века. Нептун был посещён лишь одним космическим аппаратом, <<Вояджером-2>>, который пролетел вблизи от планеты 25 августа 1989 года.


\begin{wrapfigure}{r}{0.4\textwidth}
  \vspace{-1pc}
  \center \includegraphics[width = 0.35\textwidth]{img/Neptune_diagram.png}
  \caption{Строение Нептуна.}
  \vspace{-1pc}
  \begin{note_enum}
    \item Верхняя атмосфера, верхние облака
    \item Атмосфера, состоящая из водорода, гелия и метана
    \item Мантия, состоящая из водяного, аммиачного и метанового льда
    \item Каменно--ледяное ядро
  \end{note_enum}
\end{wrapfigure}

Нептун по составу близок к Урану, и обе планеты отличаются по составу от более крупных планет-гигантов — Юпитера и Сатурна. Иногда Уран и Нептун помещают в отдельную категорию <<ледяных гигантов>>. Атмосфера Нептуна, подобно атмосфере Юпитера и Сатурна, состоит в основном из водорода и гелия. 

Атмосфера составляет примерно $10$--$20\,\%$ от общей массы планеты, и расстояние от поверхности до конца атмосферы составляет $10$--$20\,\%$ расстояния от поверхности до ядра. Вблизи ядра давление может достигать $10$ ГПа. Объёмные концентрации метана, аммиака и воды найдены в нижних слоях атмосферы. Постепенно эта более тёмная и более горячая область уплотняется в перегретую жидкую мантию, где температуры достигают $2000$--$5000$ К. Масса мантии Нептуна превышает земную в $10$--$15$ раз, по разным оценкам, и богата водой, аммиаком, метаном и прочими соединениями. По общепринятой в планетологии терминологии эту материю называют ледяной, даже при том, что это горячая, очень плотная жидкость. Эту жидкость, обладающую высокой электропроводимостью, иногда называют океаном водного аммиака. На глубине $7000$ км условия таковы, что метан разлагается на алмазные кристаллы, которые <<падают>> на ядро. Согласно одной из гипотез, имеется целый океан <<алмазной жидкости>>. В атмосфере Нептуна бушуют самые сильные ветры среди планет Солнечной системы, по некоторым оценкам, их скорости могут достигать $2100$ км/ч.

Но, как у всех гигантов Солнечной системы, у Нептуна полно собственных спутников, давайте же познакомимся с ними поближе!

\section{Спутники}

\begin{itemize}
  \item \textbf{Тритон}~--- Крупнейший спутник Нептуна Тритон был открыт английским астрономом Уильямом Ласселом в 1846 году, всего через $17$ дней после открытия планеты. Название Тритон было предложено Камиллем Фламмарионом в 1880, однако вплоть до середины XX века более употребительным было просто <<спутник Нептуна>> (второй спутник Нептуна был открыт только в 1949). Тритон~--- бог моря в греческой мифологии.
  \item \textbf{Нереида}~--- В 1949 году американско--голландский астроном Джерард Койпер открыл второй спутник Нептуна~--- Нереиду. Спутник имеет самую вытянутую орбиту из всех <<немелких>> спутников планет. Её расстояние до Нептуна меняется от $1.4$ до $9.7$ млн км. Период обращения~--- $360$ суток.
  \item \textbf{Остальные спутники}. В 1989 <<Вояджер-2>> открыл шесть спутников Нептуна. Все они движутся по круговым орбитам в прямом направлении практически в плоскости экватора планеты. Пять из них имеют периоды обращения меньше периода вращения планеты и поэтому на нептунианском небе восходят на западе и заходят на востоке; это также означает, что из-за гравитационного трения они рано или поздно упадут на Нептун. В 2002--20013 годах было открыто ещё шесть спутников Нептуна. Каждый из новооткрытых объектов имеет диаметр $30$--$60$ км и нерегулярную, вытянутую орбиту с большим наклоном. Период их обращения вокруг Нептуна составляет от 5 до 26 земных лет.
\end{itemize}

Далее вашему вниманию предоставляется таблица с большинством из известных спутников Нептуна с их подробной характеристикой (Таблицы \ref{tab:sats1} и \ref{tab:sats}).

\begin{table}[h!]

  \centering
  \begin{tabularx}{0.5\textwidth}{|C{0.1}|C{0.35}|}
    \hline
    Название & Собственное имя\\ 
    \hline
    $e$ & Эксцентриситет\\
    \hline
    $T$ & Период обращения в днях\\
    \hline
    $M$ & Масса в кг\\ 
    \hline
    Фото & Фотоснимок\\ 
    \hline
  \end{tabularx} \label{tab:sats1}
  \caption{Условные обозначения таблицы со спутниками.}
  \vspace{1pc}
  
  \begin{tabular}{|c|c|c|c|c|} 
    \hline
    Название & $e$    & $T$      & $M$             & Фото\\ 
    \hline
    Тритон   & $0.00$ & $5.8$    & $2.1\cdot10^{22}$ & \includegraphics[width = 1cm]{img/nept_sat/s1}\\ 
    \hline
    Нереида  & $0.75$ & $360.1$  & $3.1\cdot10^{19}$ & \includegraphics[width = 1cm]{img/nept_sat/s2}\\ 
    \hline
    Наяда    & $0.00$ & $0.3$    & $1.9\cdot10^{17}$ & \includegraphics[width = 1cm]{img/nept_sat/s3}\\ 
    \hline
    Таласса  & $0.00$ & $0.3$    & $3.5\cdot10^{17}$ & \includegraphics[width = 1cm]{img/nept_sat/s4}\\ 
    \hline
    Деспина  & $0.00$ & $0.3$    & $2.1\cdot10^{18}$ & \includegraphics[width = 1cm]{img/nept_sat/s5}\\ 
    \hline
    Галатея  & $0.00$ & $0.4$    & $2.1\cdot10^{18}$ & \includegraphics[width = 1cm]{img/nept_sat/s6}\\
    \hline
    Ларисса  & $0.00$ & $0.5$    & $4.9\cdot10^{18}$ & \includegraphics[width = 1cm]{img/nept_sat/s7}\\ 
    \hline
    Протей   & $0.00$ & $1.1$    & $5.0\cdot10^{19}$ & \includegraphics[width = 1cm]{img/nept_sat/s8}\\ 
    \hline
    Галимеда & $0.57$ & $1879.7$ & $9.0\cdot10^{16}$ & \includegraphics[width = 1cm]{img/nept_sat/s9}\\ 
    \hline
    Псамафа  & $0.44$ & $9115.9$ & $1.5\cdot10^{16}$ & \includegraphics[width = 1cm]{img/nept_sat/s10}\\
    \hline
    Сао      & $0.29$ & $2914.0$ & $6.7\cdot10^{16}$ & \includegraphics[width = 1cm]{img/nept_sat/s11}\\
    \hline
    Лаомедея & $0.42$ & $3167.8$ & $5.8\cdot10^{16}$ &\\ 
    \hline
    Несо     & $0.49$ & $9374$   & $1.7\cdot10^{17}$ &\\
    \hline
  \end{tabular} \label{tab:sats}
  
  \caption{Спутники Нептуна.}
\end{table}

Вы думали тут будет продолжение в виде астрономии? Но это я, интегральчик!

\section{Интегрирование}

\begin{equation} \label{eq:int}
f(x) = \begin{cases}
  x + \pi, & x \leq -\pi;\\
  -\sin x, & -\pi < x \leq 0;\\
  -x, & 0 < x \leq 1;\\
  -1 -\sin (x - 1), & 1 < x \leq 1 + \pi;\\
  x - \pi, & x > 1 + \pi.
\end{cases}
\end{equation}

Найдем производную \eqref{eq:int} в точке $x_0$. Используя те факты, что $(x)' = 1$, $(\sin x)' = \cos x$, с правилами взятия производной сложной функции \big($[f \circ g]' = f'(g(x)) \cdot g'(x)$\big), а также с осознанием того, что функция дифференцируемая на $\mathbb{R}$, найдем производную в зависимости от $x_0$:

\begin{equation} \label{eq:int_der}
f'(x_0) = \begin{cases}
  1, & x_0 \leq -\pi;\\
  -\cos x_0, & -\pi < x_0 \leq 0;\\
  -1, & 0 < x_0 \leq 1;\\
  -\cos (x_0 - 1), & 1 < x_0 \leq 1 + \pi;\\
  1, & x_0 > 1 + \pi.
\end{cases}
\end{equation}

Пусть нам дан отрезок $[a;\,b]$, пересечение которого с каждым их промежутков, имеет меру Лебега больше нуля. Проинтегрируем функцию на этом отрезке.

\begin{multline}
  \int \limits_a^b f(x)\,dx = \sum \limits_{i = 1}^5 \int \limits_{a_i}^{b_i} f_i(x)\,dx = \\
  = \int \limits_a^{-\pi}(x + \pi)\,dx + \int \limits_{-\pi}^{0}(-\sin x)\,dx + \int \limits_0^1 (-x)\,dx + \int \limits_1^{1 + \pi} \big(-1 - \sin(x - 1) \big)\,dx + \int \limits_{1 + \pi}^b (x - \pi)\,dx = \\
  = \left.\left[\frac{x^2}{2} + \pi \cdot x \right]\right|_a^{-\pi} 
  + \left.\cos x\right|_{-\pi}^{0}
  - \left.\frac{x^2}{2} \right|_{0}^{1}
  - \left.\left[x - \cos(x - 1) \right]\right|_{1}^{1 + \pi}
  + \left.\left[\frac{x^2}{2} - \pi \cdot x \right]\right|_{1 + \pi}^{b} =\\
  = \frac{\pi^2 - a^2}{2} + \pi(a - \pi) + 2 - \frac{1}{2} - \pi - 2 + \frac{b^2 - (1 + \pi)^2}{2} - \pi(b - 1 - \pi) =\\
  = \frac{1}{2} \cdot \left( -a^2 + 2\pi a + b^2 - 2\pi b + 2 \right)
\end{multline}

Ох, как же я устал интегрировать, нужно чего-то интересное, чтобы затмить эту усталость\dots~ Возможно, теперь стоит поговорить о затмениях?

\section{Затмения}

Затмение~--- астрономическая ситуация, при которой одно небесное тело заслоняет свет от другого небесного тела. Наиболее известны лунные и солнечные затмения. Также существуют такие явления, как прохождения планет (Меркурия и Венеры) по диску Солнца. Лунное затмение наступает, когда Луна входит в конус тени, отбрасываемой Землёй. Диаметр пятна тени Земли на расстоянии 363\,000 км (минимальное расстояние Луны от Земли) составляет около 2.5 диаметров Луны, поэтому Луна может быть затенена целиком. Солнечное затмение происходит, когда Луна попадает между наблюдателем и Солнцем, и загораживает его. Поскольку Луна перед затмением обращена к нам неосвещённой стороной, то перед затмением всегда бывает новолуние, то есть Луна не видна. Создаётся впечатление, что Солнце закрывается чёрным диском; наблюдающий с Земли видит это явление как солнечное затмение.

А теперь посмотрим на различные типы затмений (на следующей странице).

\newpage
\begin{figure}
  
  \begin{subfigure}[b]{0.45\textwidth}
    \includegraphics[width = \textwidth]{img/eclipses/1}
    \caption{Частичное затмение}  
  \end{subfigure}
  \centering
  \begin{subfigure}[b]{0.45\textwidth}
    \includegraphics[width = \textwidth]{img/eclipses/2}
    \caption{Полное солнечное затмение}  
  \end{subfigure}
  
  \begin{subfigure}[b]{0.45\textwidth}
    \includegraphics[width = \textwidth]{img/eclipses/3}
    \caption{Кольцеобразное затмение}  
  \end{subfigure}
  \centering
  \begin{subfigure}[b]{0.45\textwidth}
    \includegraphics[width = \textwidth]{img/eclipses/4}
    \caption{Гибридное затмение}  
  \end{subfigure}
  
  \caption{Виды затмений}
\end{figure}

\end{document}
