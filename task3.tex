\documentclass[12pt]{article}

\usepackage[utf8]{inputenc}
\usepackage[russian]{babel}
\usepackage[margin=1cm, paper=a4paper]{geometry}
\usepackage{amsthm,mathtools,amsmath,amsfonts}
\usepackage{framed}
\usepackage{gensymb}
\usepackager{hyperref}

\author{Мумладзе Максимелиан}
\title{Домашнее задание №2}
\date{\today}

\DeclareMathOperator{\mat}{Mat}

\newtheorem{thm}{Теорема}[section]
\theoremstyle{definition}\newtheorem{defi}{Определение}

\newcounter{taskn}[section]
\newenvironment{task}[1]{%
\begin{framed}
  \noindent
  \underline{\bfseries Задача \stepcounter{taskn}\thetaskn. <<#1>>}
  \setlength{\parindent}{0cm}
}{%
\end{framed}
}

\begin{document}
\maketitle

\section{Краткий обзор}
\begin{center}\Huge

Нептун.

\end{center}

\textbf{Нептун}~--- восьмая и самая дальняя от Земли планета Солнечной системы. Нептун также является четвёртой по диаметру и третьей по массе планетой. Масса Нептуна в $17,2$ раза, а диаметр экватора в $3,9$ раза больше земных. Название планеты было предложено русско--немецким астрономом Василием Струве\footnote{\href{http://articles.adsabs.harvard.edu/full/seri/AN.../0025/0000164.000.html}{пруф}}.  Её астрономический символ  ~--- стилизованная версия трезубца Нептуна.


\end{document}