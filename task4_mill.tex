\documentclass[12pt]{article}

\usepackage[utf8]{inputenc}
\usepackage[russian]{babel}
\usepackage[margin=0.5cm, paper=a4paper]{geometry}
\usepackage{amsthm,amsmath,amsfonts}
\usepackage{mathtools}
\usepackage{framed}
\usepackage{gensymb}
\usepackage{hyperref}
\usepackage{marvosym} % For Neptune.
\usepackage{wrapfig}
\usepackage{tabularx}
\usepackage{subcaption}
\usepackage[margin=10pt,font=small,labelfont=bf,labelsep=endash]{caption}
\usepackage{enumitem}
\usepackage{graphicx}
\usepackage{tikz}
\usepackage{xcolor}

\author{Мумладзе Максимелиан}
\title{Домашнее задание №3}
\date{\today}
\pagenumbering{gobble}

\DeclareMathOperator{\mat}{Mat}


\newtheorem{thm}{Теорема}[section]
\theoremstyle{definition}\newtheorem{defi}{Определение}


\newcounter{taskn}[section]
\newenvironment{task}[1]{%
\begin{framed}
  \noindent
  \underline{\bfseries Задача \stepcounter{taskn}\thetaskn. <<#1>>}
  \setlength{\parindent}{0cm}
}{%
\end{framed}
}

\newenvironment{note_enum}{%
\begin{framed}
  
  \begin{small}
  \begin{enumerate}[wide, labelwidth=!, labelindent=0pt]
    \vspace{-1pc}
}{%
  \end{enumerate}
  \end{small}
  \vspace{-1pc}
\end{framed}
}


\newcolumntype{L}[1]{>{\hsize=#1\textwidth%
\raggedright\arraybackslash}X}%
\newcolumntype{R}[1]{>{\hsize=#1\textwidth%
\raggedleft\arraybackslash}X}%
\newcolumntype{C}[1]{>{\hsize=#1\textwidth%
\centering\arraybackslash}X}

%- - - - - - - - - - - - - - - - - - - - - - - - - - - - - - - - - - - - - - -
\begin{document}

\centering
\begin{tikzpicture}
  \foreach \x in {0, 0.1, ..., 19} {
    \draw[very thin, color={rgb:orange,1;white,4}] (\x, 0) -- (\x, -28);
  };
  \foreach \y in {0, 0.1, ..., 28} {
    \draw[very thin, color={rgb:orange,1;white,4}] (0, -\y) -- (19, -\y);
  };
  
  \foreach \x in {0, 0.5, ..., 19} {
    \draw[thick, color={rgb:brown,1;white,2}] (\x, 0) -- (\x, -28);
  };
  \foreach \y in {0, 0.5, ..., 28} {
    \draw[thick, color={rgb:brown,1;white,2}] (0, -\y) -- (19, -\y);
  };
  
  \foreach \x in {0, 1, ..., 19} {
    \draw[very thick, color=brown] (\x, 0) -- (\x, -28);
  };
  \foreach \y in {0, 1, ..., 28} {
    \draw[very thick, color=brown] (0, -\y) -- (19, -\y);
  };
\end{tikzpicture}

\end{document}