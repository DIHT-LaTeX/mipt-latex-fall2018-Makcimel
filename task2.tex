\documentclass[11pt]{article}

\usepackage[utf8]{inputenc}
\usepackage[russian]{babel}
\usepackage[margin=1cm, paper=a5paper]{geometry}
\usepackage{amsthm,mathtools,amsmath,amsfonts}
\usepackage{framed}
\usepackage{gensymb}

\author{Мумладзе Максимелиан}
\title{Домашнее задание №2}
\date{\today}
\pagenumbering{gobble}

\DeclareMathOperator{\mat}{Mat}

\newtheorem{thm}{Теорема}[section]
\theoremstyle{definition}\newtheorem{defi}{Определение}

\newcounter{taskn}[section]

\newenvironment{task}[1]{%
\begin{framed}
  \noindent
  \underline{\bfseries Задача~\stepcounter{taskn}\thetaskn. <<#1>>}
  \setlength{\parindent}{0cm}
}{%
\end{framed}
}

\begin{document}
\maketitle

\part{Конспект по теории вероятностей}

\section{Центральная предельная теорема}

\begin{thm}[Линдеберга]

Пусть $\{\xi_k\}_{k \geq 1}$~--- независимая случайная величина, $\mathsf{E} \xi_k^2 < +\infty ~ \forall k$, обозначим $m_k = \mathsf{E} \xi_k,$ $\delta_k^2 = \mathsf{D} \xi_k > 0,$ $S_n = \sum\limits_{i = 1}^n \xi_k;$ $D_n^2 = \sum\limits_{k = 1}^n \delta_k^2$ и $F_k(x)$~--- функция распределения $\xi_k$. Пусть выполняется условие Линдеберга, то есть

$$
\forall \varepsilon > 0 ~ \frac{1}{\mathsf{D}_n^2} \sum\limits_{k = 1}^n \int\limits_{\big\{x: |x - m_k| > \varepsilon \mathsf{D}_n\big\}} (x - m_k)^2\,dF_k(x) \xrightarrow[m \rightarrow \infty]{} 0.
$$

Тогда $\frac{S_n - \mathsf{E} S_n}{\sqrt{\mathsf{D} S_n}} \xrightarrow{d} \mathcal{N}(0, 1), n \rightarrow \infty$.

\end{thm}


\section{Гауссовские случайные векторы}

\begin{defi}

Случайный вектор $\vec \xi \sim \mathcal{N}(m, \Sigma)$~--- гауссовский, если его характеристическая функция $\varphi \xi(\vec t\,) = \exp \big(i(\vec m, \vec t\,) - \frac{1}{2} (\Sigma \vec t, \vec t\,) \big), \vec m \in \mathbb{R}^n$, $\Sigma$~--- симметрично неотрицательно определенная матрица.

\end{defi}

\begin{defi}

Случайный вектор $\vec \xi$~--- гауссовский, если он представляется в следующем виде: $\vec \xi = A \vec \eta + \vec b$, где $\vec b \in \mathbb{R}^n, A \in \mat(n \times m)$ и $\vec \eta = (\eta_1, \dots, \eta_m)$~--- независимый и $\sim \mathcal{N}(0, 1)$.

\end{defi}

\begin{defi}

Случайный вектор $\vec \xi$~--- гауссовский, если $\forall \lambda \in \mathbb{R}^n$ случайный вектор $(\vec \lambda, \vec \xi\,)$ имеет нормальное распределение.

\end{defi}

\begin{thm}[Об эквивалентности определений гауссовских векторов]

Предоставлено три определения эквивалентности.

\end{thm}


\part{Задачи}

\section{Задачи по астрономии}

\begin{task}{Бейрут}

В какой момент по истинному солнечному времени $1$ сентября Регул ($\alpha_1 = 10^\text{h} \, 9^\text{m}, \delta_1 = 11\degree \, 53'$) и Шертан ($\alpha_2 = 11^\text{h} \, 15^\text{m}, \delta_2 = 15 \degree \, 20'$) находятся на одном альмукантарате в Бейруте ($\varphi = 33 \degree \, 53'$)?

\end{task}

\begin{task}{К Сатурну!}

Космический корабль запустили с поверхности Земли к Сатурну по наиболее энергетически выгодной траектории. При движении по орбите корабль пролетел мимо астероида--троянца ($624$) Гектор.\\[10pt]
Определите большую полуось и эксцентриситет полученной орбиты, скорость старта с поверхности Земли, а также угол между направлением на Солнце и на Сатурн в момент старта корабля. Орбиты планет считать круговыми. Оцените относительную скорость корабля и астероида в момент сближения.

\end{task}

\begin{task}{Dark Matters}

В некотором скоплении галактик содержится $70$ спиральных и $30$ эллиптических галактик. Известно, что абсолютная величина галактик равна $-20$, соотношение масса--светимость составляет $15 \, \mathfrak{M}_\odot / L_\odot$. У спиральных галактик в данном скоплении максимальная скорость вращения составляет $210$ км/c, соотношение масса--светимость~--- $5 \, \mathfrak{M}_\odot / L_\odot$.\\[10pt]
Оцените долю тёмной материи внутри скопления, если масса межгалактического газа на порядок превышает массу галактик, а типичные скорости галактик составляют $1000$ км/c. Размер скопления составляет $7$ Мпк. Абсолютная звёздная величина Млечного Пути~--- $-20.9$.

\end{task}

\begin{task}{Обратный комптон--эффект}

Обратным эффектом Комптона (ОЭК) называют явление рассеяния фотона на ультрарелятивистском свободном электроне свободном электроне, при котором происходит перенос энергии от электрона к фотону. Рассмотрите ОЭК для фотонов реликтового излучения. При какой энергии электронов в направленном пучке рассеянное излучение можно будет зарегистрировать на фотоприёмнике?

\end{task}


\part{И еще кое-что\dots}

\section{Отзыв}

\begin{itemize}

\item[$\flat$] Отличный курс.
\item[$\aleph$] Организация на высшем уровне. Единственное~--- хотелось бы, чтобы занятия проводились только в одной аудитории, постоянно не переезжая.
\item[$\Im$] Качество материала тоже очень хорошее. Видно, что человек усердно работает.
\item[$\odot$] То же самое про преподношение.

\end{itemize}

\end{document}