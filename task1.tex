\documentclass[12pt]{article}

\usepackage[T2A]{fontenc}
\usepackage[utf8]{inputenc}

\usepackage[russian]{babel}

\title{Домашняя работа по \LaTeX ~№1}
\author{Максимелиан Мумладзе}
\date{}

% - - - - - - - - - - - - - - - - - - - - - - - - - - - 

\begin{document}
\maketitle

\begin{flushright}
\hfill\textit{Audi multa,\\loquere pauca}
\end{flushright}

\vspace{20pt}

Это мой первый\iffalse в кольце вычета $\approx 20$)\fi документ в системе компьютерной вёрстки \LaTeX.

\begin{center}
\Huge{\sffamily <<Ура!!!>>}
\end{center}

А теперь формулы. \textsc{Формула}~--- краткое и точное словесное выражение, определение или же ряд математических величин, выраженный условными знаками.

\vspace{15pt}

\hspace{28pt}{\large\bfseries Термодинамика}

Уравнение Менделеева--Клапейрона~--- уравнение состояния идеального газа, имеющее вид $pV = \nu RT,$ где $p$~--- давление, $V$~--- объем, занимаемый газом, $T$~--- температура газа, $\nu$~--- количество вещества газа, а $R$~--- универсальная газовая постоянная.

\vspace{15pt}

\hspace{28pt}{\large\bfseries Геометрия \hfill Планиметрия}

Для \textsl{плоского} треугольника со сторонами $a, b, c$ и углом $\alpha$, лежащем против стороны $a$, справедливо соотношение

$$
a^2 = b^2 + c^2 - 2 b c \cos\alpha,
$$

из которого можно выразить косинус угла треугольника:

$$
\cos\alpha = \frac{b^2 + c^2 - a^2}{2bc}.
$$\\

Пусть $p$~--- полупериметр треугольника, тогда путем несложных преобразований можно получить, что 

$$
\tg\frac{\alpha}{2} = \sqrt{\frac{(p - b)(p - c)}{p(p - a)}},
$$

\vspace{1cm}

На сегодня, пожалуй, хватит\dots ~Удачи!

\end{document}